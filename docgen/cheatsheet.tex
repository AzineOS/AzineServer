\documentclass[10pt,landscape]{article}
\usepackage{multicol}
\usepackage{calc}
\usepackage{ifthen}
\usepackage[landscape]{geometry}
\usepackage{amsmath,amsthm,amsfonts,amssymb}
\usepackage{color,graphicx,overpic}
\usepackage{hyperref}
\usepackage[utf8]{inputenc}
\usepackage{latexsym}

\pdfinfo{
  /Title Arcan LUA API cheat sheet
  /Creator (Tex)
  /Producer (pdfTeX 1.40.0)
  /Author (Björn Ståhl)
  /Subject (Computing)
  /Keywords (api, visualization, frontend)}

% This sets page margins to .5 inch if using letter paper, and to 1cm
% if using A4 paper. (This probably isn't strictly necessary.)
% If using another size paper, use default 1cm margins.
\ifthenelse{\lengthtest { \paperwidth = 11in}}
    { \geometry{top=.5in,left=.5in,right=.5in,bottom=.5in} }
    {\ifthenelse{ \lengthtest{ \paperwidth = 297mm}}
        {\geometry{top=1cm,left=1cm,right=1cm,bottom=1cm} }
        {\geometry{top=1cm,left=1cm,right=1cm,bottom=1cm} }
    }

% Turn off header and footer
\pagestyle{empty}

% Redefine section commands to use less space
\makeatletter
\renewcommand{\section}{\@startsection{section}{1}{0mm}%
                                {-1ex plus -.5ex minus -.2ex}%
                                {0.5ex plus .2ex}%x
                                {\normalfont\large\bfseries}}
\renewcommand{\subsection}{\@startsection{subsection}{2}{0mm}%
                                {-1explus -.5ex minus -.2ex}%
                                {0.5ex plus .2ex}%
                                {\normalfont\normalsize\bfseries}}
\renewcommand{\subsubsection}{\@startsection{subsubsection}{3}{0mm}%
                                {-1ex plus -.5ex minus -.2ex}%
                                {1ex plus .2ex}%
                                {\normalfont\small\bfseries}}
\makeatother

% Define BibTeX command
\def\BibTeX{{\rm B\kern-.05em{\sc i\kern-.025em b}\kern-.08em
    T\kern-.1667em\lower.7ex\hbox{E}\kern-.125emX}}

% Don't print section numbers
\setcounter{secnumdepth}{0}
\setlength{\parindent}{0pt}
\setlength{\parskip}{0pt plus 0.5ex}

%My Environments
\newtheorem{example}[section]{Example}
% -----------------------------------------------------------------------

\begin{document}
\raggedright
\footnotesize
\begin{multicols}{3}

% multicol parameters
% These lengths are set only within the two main columns
%\setlength{\columnseprule}{0.25pt}
\setlength{\premulticols}{1pt}
\setlength{\postmulticols}{1pt}
\setlength{\multicolsep}{1pt}
\setlength{\columnsep}{2pt}

\begin{center}
     \Large{\underline{ARCAN LUA API}} \
\end{center}

\section{Entry points}
	\emph{All entry-points are prefixed with themename\_}.\\

	\textbf{clock\_pulse}\emph{()} \emph{invoked at a fixed rate (default: once every 25ms)}.\\
	\textbf{input}\emph{(inputtable)} \emph{see IOTable}.\\
	\textbf{video\_event}\emph{(source, argtbl)}, source:vobjid, argtbl:vidtable format \emph{resizes, completed transformations, ...}.\\
	\textbf{themename}\emph{()}	\emph{invoked directly after initializing video, audio, events, ...}.\\
	\textbf{on\_show}\emph{()} \emph{invoked after themename} (deprecated).\\

	\emph{All functions prefixed with * are considered experimental.}\\
\section{Resource}
		\textbf{resource}\emph{(name : searchmask(THEME\_RESOURCE, SHARED\_RESOURCE))} $\rightarrow$ (exists ? true or false).\\
		\textbf{glob\_resource}\emph{searchmask, tgtmask (THEME, SHARED, ALL)} $\rightarrow$ resourcetbl.\\
		\textbf{zap\_resource}\emph{(name)} $\rightarrow$ true or false, locates and deletes the referenced resource (in theme only).\\
		\textbf{open\_rawresource}\emph{(resource)} $\rightarrow$ open resource for write (only one open allowed, in theme only).\\
		\textbf{write\_rawresource}\emph{(line)}, write line to opened rawresource.\\
		\textbf{close\_rawresource}\emph{()}, close/flush any opened rawresource.\\
		
\section{Target}
		\textbf{launch\_target}\emph{(gametitle, launchmode)}, LAUNCH\_INTERNAL ? $\rightarrow$ vid, aid.\\
		\textbf{target\_input}\emph{(inputtable)}, inject an event into the running target.\\
		\textbf{suspend\_target}\emph{(tgtvid)}, attempt to suspend target.\\
		\textbf{resume\_target}\emph{(tgtvid)}, attempt to resume target.\\

\section{System}
	\textbf{kbd\_repeat}\emph{(rate (ms))} $\rightarrow$ enable keyboard I/O generating repeating events.\\
	\textbf{system\_load}\emph{(resource)} $\rightarrow$ load script and return function reference.\\
	\textbf{shutdown}\emph{()} $\rightarrow$ kill targets, unload resources and shutdown.\\
	\textbf{switch\_default\_scalemode}\emph{(enumint)} $\rightarrow$ override the default or command-line set scalemode.\\
	\textbf{switch\_default\_texmode}\emph{(enumint)} $\rightarrow$ override the default texture coordinate mode.\\
	\textbf{switch\_default\_imageproc}\emph{(enumimgproc)} $\rightarrow$ override the default image-loading preprocessor.\\
	\textbf{warning}\emph{(msg)} $\rightarrow$ push warning as a string to the device associated with ERR.\\
	
\section{Database}
	\textbf{store\_key}\emph{(key, value)} $\rightarrow$ store key/value (strings) in database as theme-specific table.\\
	\textbf{get\_key}\emph{(key)} $\rightarrow$ get value (or nil) from theme-specific database table.\\
	\textbf{game\_cmdline}\emph{(title)} $\rightarrow$ generate execstr for a specific game.\\
	\textbf{list\_games}\emph{(filtertbl)} $\rightarrow$ query database for games matching filter, empty table for all.\\
	\textbf{list\_targets}\emph{()} $\rightarrow$ get a list of all targets.\\
	\textbf{game\_info}\emph{(title)} $\rightarrow$ get a gametable for title (or nil)\\
	\textbf{game\_family}\emph{(title)} $\rightarrow$ get the family for a title (or nil)\\
	\textbf{game\_genres}\emph{()} $\rightarrow$ list all distinct (unique) genres / subgenres.\\

\section{Audio}
	\textbf{stream\_audio}\emph{(resource)} $\rightarrow$ returns an aid for a stream (or nil)\\
	\textbf{play\_audio}\emph{(aid)} $\rightarrow$ start decode/buffer/playback for a stream\\
	\textbf{pause\_audio}\emph{(aid)} $\rightarrow$ try and pause (slightly unstable) audio stream playback.\\
	\textbf{delete\_audio}\emph{(aid)} $\rightarrow$ deallocate and stop playback (slight delay due to buffering).\\
	\textbf{load\_asample}\emph{(resource)} $\rightarrow$ load sample (wav) into aid.\\
	\textbf{audio\_gain}\emph{(aid, newgain (0..1), [time])} $\rightarrow$ set or fade gain.\\

\section{Video}
	\textbf{load\_image}\emph{(resource, [initial zval (0..255)])} $\rightarrow$ vid\\
	\textbf{load\_image\_asynch}\emph{(resour
	\textbf{delete\_image}\emph{(vid)} $\rightarrow$ immediately delete and deallocate (won't emit event).\\
	\textbf{show\_image}\emph{(vid)} $\rightarrow$ wrapper for blend\_image(vid, 1.0).\\
	\textbf{hide\_image}\emph{(vid)} $\rightarrow$ wrapper for blend\_image(vid, 0.0).\\
	\textbf{move\_image}\emph{(vid, absx (px), absy (px), [time])} $\rightarrow$ reposition to absolute coordinates (absx, absy).\\
	\textbf{rotate\_image}\emph{(vid, absangz, [time])} $\rightarrow$.\\
	\textbf{scale\_image}\emph{(vid, xfact, yfact, [time])} $\rightarrow$ relative scale (1.0 == initial size), xfact or yfact zero = force aspect.\\
	\textbf{resize\_image}\emph{(vid, width (px), height (px), [time])} $\rightarrow$ absolute resize, width or height zero = force aspect.\\
	\textbf{blend\_image}\emph{(vid, opacity (0..1), [time])}\\
	\textbf{order\_image}\emph{(vid, newz)} $\rightarrow$ 0 = further back, 255 = foreground.\\
	\textbf{instance\_image}\emph{(vid)} $\rightarrow$ clone the vid (share resources, child dies if vid dies), returns newvid.\\
	\textbf{*link\_image}\emph{(vid, parent)} $\rightarrow$ bind vid to parent coordinate system.\\
	\textbf{*set\_image\_as\_frame}\emph{dstvid, srcvid, frameno} $\rightarrow$ dstvid will have srcvid on frameno.\\
	\textbf{image\_framesetsize}\emph{vid, ncells} $\rightarrow$ set the number of allowed frames (0 < allowed < 256) on vid.\\
	\textbf{image\_active\_frame}\emph{vid, frameno} $\rightarrow$ change the currently active frame (reset to default if slot unset).\\
	\textbf{expire\_image}\emph{(vid, lifetime)} $\rightarrow$ invoke delete\_image after lifetime ticks, emits event.\\
	\textbf{reset\_image\_transform}\emph{(vid)} $\rightarrow$ remove all queued transformations.\\
	\textbf{instant\_image\_transform}\emph{(vid)} $\rightarrow$ force all transformations, regardless of time.\\
	\textbf{image\_mask\_toggle}\emph{(vid, enumint)} $\rightarrow$ toggle relative value lookup for a specific property.\\
	\textbf{image\_mask\_set}\emph{(vid, enumint)}\\
	\textbf{image\_mask\_clear}\emph{(vid, enumint)}\\
	\textbf{image\_surface\_properties}\emph{(vid, [time])} $\rightarrow$ returns current (or future) surface properties as a surftbl\\
	\textbf{image\_surface\_initial\_properties}\emph{(vid)} $\rightarrow$ returns initial (at load time) surface properties as a surftbl\\
	\textbf{*image\_program}\emph{(vid, vertprog, fragprog)} $\rightarrow$ load and associate a shader (GPU program).\\
	\textbf{render\_text}\emph{(formatstr)} $\rightarrow$ returns vid and table of lineheights.\\
	\textbf{fill\_surface}\emph{(width (px), height (px), r (0..255), g, b)} $\rightarrow$ generate a single-colored vid.\\
	\textbf{force\_image\_blend}\emph{(vid, boolint)} $\rightarrow$ always alpha-blend (transparent images).\\
	\textbf{*push\_video\_context}\emph{()} $\rightarrow$ add to context stack (if there's any free stack slots).\\
	\textbf{*pop\_video\_context}\emph{()} $\rightarrow$ deallocate current context (won't emit events) and load pushed context.\\

\section{Frameserver}
	\textbf{play\_movie}\emph{(vid)} $\rightarrow$ start playback of preloaded movie.\\
	\textbf{load\_movie}\emph{(resource, [loop])} $\rightarrow$ launch frameserver, return vid,aid. \\
	\textbf{pause\_movie}\emph{(vid)} $\rightarrow$ unreliable (due to openAL). \\
	\textbf{resume\_movie}\emph{(vid)} $\rightarrow$ unreliable (due to openAL). \\
	
\section{Collision / Picking}
	\textbf{image\_hit}\emph{(vid, x, y)} $\rightarrow$ boolnum if x, y is on pickable image vid.\\
	\textbf{pick\_items}\emph{(x, y)} $\rightarrow$ returns ary of vids at x, y.\\

\section{LED}
	\textbf{set\_led}\emph{(ctrl, led, state)} $\rightarrow$ toggle led on or off.\\
	\textbf{*led\_intensity}\emph{(ctrl, led, val)} $\rightarrow$ untested, for advanced led controllers.\\
	\textbf{*set\_led\_rgb}\emph{(ctrl, led, rv, gv, bv)} $\rightarrow$ untested, for advanced led controllers.\\
	\textbf{controller\_leds}\emph{(ctrl)} $\rightarrow$ number of leds associated with a controller.\\

\section{IMGTable format}
	\emph{width}, \emph{height}, \emph{x}, \emph{y}, \emph{angle}, \emph{opacity}

\section{GameTable format}
    \emph{gameid}, \emph{targetid}, \emph{title}, \emph{genre}, \emph{subgenre}, \emph{setname}, \emph{buttons}, \emph{manufacturer}, \emph{players}, \emph{input}, \emph{year}

\section{IOTable format}
    kind :- analog, digital\\
	(kind == analog), devid, subid, source (mouse, joystick, ...)\\
	(kind == digital), translated (bool), active (bool), devid, subid\\
	(kind == digital, translated:true) number, keysym, modifiers\\ 

\section{GameFilter}
	fill the table with desired options; \\
	year (number or 0), \\
	title (string), \\
	genre (string), \\
	subgenre (string), \\
	players (number), \\
	buttons (number) \\	
\end{multicols}
\pagebreak

\section{Global Variables}
	\begin{itemize}
		\item VRESW (num) set to the window/display width.
		\item VRESH (num) set to the window/display height.
		\item VCTXVIDLIMIT (num) vid stack context size.
		\item WORLDID (num) \emph{vid} of a special object that refers to the outmost coordinate system.
		\item BADID (num) functions returning a vid can also return BADID if the function failed for some reason.
		\item CLOCK (num) built-in clock rate (default, 25hz).
		\item JOYSTICKS (num) number of detected (and opened) joysticks.
		\item LEDCONTROLLERS (num) number of detected LED controllers.
		\item THEME\_RESOURCE (num) mask value for specifying themepath/themename/ as a search-path for resources.
		\item SHARED\_RESOURCE (num) mask value for specifying resourcepath as a search-path for resources.
		\item API\_VERSION\_MAJOR (num)
		\item API\_VERSION\_MINOR (num)
		\item LAUNCH\_EXTERNAL (num) mask value for specifying that the target should be launched in external mode.
		\item LAUNCH\_INTERNAL (num) mask value for specifying that the target should be launched in internal mode.
		\item MASK\_ORIENTATION, MASK\_OPACITY, MASK\_POSITION, MASK\_SCALE, MASK\_UNPICKABLE (num)
		\item THEMENAME (text) name of the currently loaded theme.
		\item RESOURCEPATH, THEMEPATH, BINPATH, LIBPATH (text)
		\item INTERNALMODE (text) estimated level of internal launch support.
		\item NOW (num) used to specify that a transformation (move, scale, blend, ...) should happen immediately.
		\item TEX\_REPEAT (num) see switch\_default\_texmode.
		\item TEX\_CLAMP (num) see switch\_default\_texmode.
		\item SCALE\_NOPOW2 (num) see switch\_default\_scalemode.
		\item SCALE\_TXCOORD (num) see switch\_default\_scalemode.
		\item SCALE\_POW2 (num) see switch\_default\_scalemode.
	\end{itemize}
\pagebreak

\section{Planned Features}
There's no strict schedule for these features currently. A lot will depend on community interest but the main priority
is stability for the currently available API / feature-set. 

	\subsection{Input}
	\begin{itemize}
        \item (RGB+intensity) LED controllers (don't have access to any currently).
		\item Sample-buffer and rate-limiter for emitting axis events to LUA (lessen impact of high-resolution mice etc).
		\item scripts to convert from keyconfig scripts to mame etc.
        \item Manymice support (includes the internal target launcher).
		\item Wiimote support.
		\item Touchscreen support.
	\end{itemize}

	\subsection{System}
	\begin{itemize}
		\item Replace IPC- esque commands with event- serialization using ProtocolBuffers.
		\item Yield alternative for SDL\_delay in framequeue on EAGAIN.
		\item Android port (don't have any decent GLES tablet atm.)
		\item Support- library for cooperative internal-launch for more efficient / less aggressive internal-launch, pause- functions etc. with possibly patches for mplayer, mame, mess, ...
		\item Collision- / Intersection- tests for VIDs, basic 2d physics?
	\end{itemize}

    \subsection{OSX}
    \begin{itemize}
		\item Seems like a priority inversion problem when going into fullscreen / full-windowed mode, makes internal launch / movies slow.
		\item Full- screen options don't work at all on Lion (SDL- problem as well, patch apparently on its way).
		\item Focus isn't properly moved between main window / external target and back (should use launch- hidden).
		\item Internal launch- mode drops focus (also launch-hidden )
		\item Overall "design" does not fit well with 'Bundle' style.
    \end{itemize}
	
	\subsection{Windows}
	\begin{itemize}
		\item Internal launch-mode missing.
	\end{itemize}
	
	\subsection{Video}
	\begin{itemize}
		\item Direct mng support as animation format alternative to multiple single image files.
		\item Two-sided objects (requires orientation + multiframe + updated picking).
		\item More advanced frameserver support (webcam, ...).
		\item Flag to make part of a render string outlined
		\item Stream-export vid+aid (or world) to ffmpeg as external encoder / frameserver.
		\item \emph{halign} (centered, left, right) and \emph{valign} (top, middle, bottom with multiline-multisize) text support.
	\end{itemize}
	
	\subsection{Audio}
		\begin{itemize}
			\item slightly more format support, perhaps move to an audio\_frameserver.
			\item mpd client support.
			\item speex chat.
			\item scriptable synthesis.
		\end{itemize}
		
	\subsection{Database}
	\begin{itemize}
		\item Target filters
		\item Game-/Target- based key/value tagging (current is only theme based)
		\item Launch history.
		\item Support-scripts for movies / tv-series.
	\end{itemize}
	
	\subsection{LUA}
	\begin{itemize}
		\item Replace it with Squirrel... 
		\item Better scheduling / preemption of lua- execution. Event multiplexing.
		\item Working sandboxing, currently there's directory traversal in resource calls.
		\item Restricted online features (gamelists, highscores, playcounts, ...)
		\item Structure for specifying theme requirements (resolution, aspect-ratio, ...)
		\item Nudge relative movement convenience function
	\end{itemize}
	
% You can even have references
\rule{0.3\linewidth}{0.25pt}
\scriptsize

\end{document}

